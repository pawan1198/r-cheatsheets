% Options for packages loaded elsewhere
\PassOptionsToPackage{unicode}{hyperref}
\PassOptionsToPackage{hyphens}{url}
%
\documentclass[
]{article}
\usepackage{amsmath,amssymb}
\usepackage{iftex}
\ifPDFTeX
  \usepackage[T1]{fontenc}
  \usepackage[utf8]{inputenc}
  \usepackage{textcomp} % provide euro and other symbols
\else % if luatex or xetex
  \usepackage{unicode-math} % this also loads fontspec
  \defaultfontfeatures{Scale=MatchLowercase}
  \defaultfontfeatures[\rmfamily]{Ligatures=TeX,Scale=1}
\fi
\usepackage{lmodern}
\ifPDFTeX\else
  % xetex/luatex font selection
\fi
% Use upquote if available, for straight quotes in verbatim environments
\IfFileExists{upquote.sty}{\usepackage{upquote}}{}
\IfFileExists{microtype.sty}{% use microtype if available
  \usepackage[]{microtype}
  \UseMicrotypeSet[protrusion]{basicmath} % disable protrusion for tt fonts
}{}
\makeatletter
\@ifundefined{KOMAClassName}{% if non-KOMA class
  \IfFileExists{parskip.sty}{%
    \usepackage{parskip}
  }{% else
    \setlength{\parindent}{0pt}
    \setlength{\parskip}{6pt plus 2pt minus 1pt}}
}{% if KOMA class
  \KOMAoptions{parskip=half}}
\makeatother
\usepackage{xcolor}
\usepackage[margin=1in]{geometry}
\usepackage{color}
\usepackage{fancyvrb}
\newcommand{\VerbBar}{|}
\newcommand{\VERB}{\Verb[commandchars=\\\{\}]}
\DefineVerbatimEnvironment{Highlighting}{Verbatim}{commandchars=\\\{\}}
% Add ',fontsize=\small' for more characters per line
\usepackage{framed}
\definecolor{shadecolor}{RGB}{248,248,248}
\newenvironment{Shaded}{\begin{snugshade}}{\end{snugshade}}
\newcommand{\AlertTok}[1]{\textcolor[rgb]{0.94,0.16,0.16}{#1}}
\newcommand{\AnnotationTok}[1]{\textcolor[rgb]{0.56,0.35,0.01}{\textbf{\textit{#1}}}}
\newcommand{\AttributeTok}[1]{\textcolor[rgb]{0.13,0.29,0.53}{#1}}
\newcommand{\BaseNTok}[1]{\textcolor[rgb]{0.00,0.00,0.81}{#1}}
\newcommand{\BuiltInTok}[1]{#1}
\newcommand{\CharTok}[1]{\textcolor[rgb]{0.31,0.60,0.02}{#1}}
\newcommand{\CommentTok}[1]{\textcolor[rgb]{0.56,0.35,0.01}{\textit{#1}}}
\newcommand{\CommentVarTok}[1]{\textcolor[rgb]{0.56,0.35,0.01}{\textbf{\textit{#1}}}}
\newcommand{\ConstantTok}[1]{\textcolor[rgb]{0.56,0.35,0.01}{#1}}
\newcommand{\ControlFlowTok}[1]{\textcolor[rgb]{0.13,0.29,0.53}{\textbf{#1}}}
\newcommand{\DataTypeTok}[1]{\textcolor[rgb]{0.13,0.29,0.53}{#1}}
\newcommand{\DecValTok}[1]{\textcolor[rgb]{0.00,0.00,0.81}{#1}}
\newcommand{\DocumentationTok}[1]{\textcolor[rgb]{0.56,0.35,0.01}{\textbf{\textit{#1}}}}
\newcommand{\ErrorTok}[1]{\textcolor[rgb]{0.64,0.00,0.00}{\textbf{#1}}}
\newcommand{\ExtensionTok}[1]{#1}
\newcommand{\FloatTok}[1]{\textcolor[rgb]{0.00,0.00,0.81}{#1}}
\newcommand{\FunctionTok}[1]{\textcolor[rgb]{0.13,0.29,0.53}{\textbf{#1}}}
\newcommand{\ImportTok}[1]{#1}
\newcommand{\InformationTok}[1]{\textcolor[rgb]{0.56,0.35,0.01}{\textbf{\textit{#1}}}}
\newcommand{\KeywordTok}[1]{\textcolor[rgb]{0.13,0.29,0.53}{\textbf{#1}}}
\newcommand{\NormalTok}[1]{#1}
\newcommand{\OperatorTok}[1]{\textcolor[rgb]{0.81,0.36,0.00}{\textbf{#1}}}
\newcommand{\OtherTok}[1]{\textcolor[rgb]{0.56,0.35,0.01}{#1}}
\newcommand{\PreprocessorTok}[1]{\textcolor[rgb]{0.56,0.35,0.01}{\textit{#1}}}
\newcommand{\RegionMarkerTok}[1]{#1}
\newcommand{\SpecialCharTok}[1]{\textcolor[rgb]{0.81,0.36,0.00}{\textbf{#1}}}
\newcommand{\SpecialStringTok}[1]{\textcolor[rgb]{0.31,0.60,0.02}{#1}}
\newcommand{\StringTok}[1]{\textcolor[rgb]{0.31,0.60,0.02}{#1}}
\newcommand{\VariableTok}[1]{\textcolor[rgb]{0.00,0.00,0.00}{#1}}
\newcommand{\VerbatimStringTok}[1]{\textcolor[rgb]{0.31,0.60,0.02}{#1}}
\newcommand{\WarningTok}[1]{\textcolor[rgb]{0.56,0.35,0.01}{\textbf{\textit{#1}}}}
\usepackage{longtable,booktabs,array}
\usepackage{calc} % for calculating minipage widths
% Correct order of tables after \paragraph or \subparagraph
\usepackage{etoolbox}
\makeatletter
\patchcmd\longtable{\par}{\if@noskipsec\mbox{}\fi\par}{}{}
\makeatother
% Allow footnotes in longtable head/foot
\IfFileExists{footnotehyper.sty}{\usepackage{footnotehyper}}{\usepackage{footnote}}
\makesavenoteenv{longtable}
\usepackage{graphicx}
\makeatletter
\def\maxwidth{\ifdim\Gin@nat@width>\linewidth\linewidth\else\Gin@nat@width\fi}
\def\maxheight{\ifdim\Gin@nat@height>\textheight\textheight\else\Gin@nat@height\fi}
\makeatother
% Scale images if necessary, so that they will not overflow the page
% margins by default, and it is still possible to overwrite the defaults
% using explicit options in \includegraphics[width, height, ...]{}
\setkeys{Gin}{width=\maxwidth,height=\maxheight,keepaspectratio}
% Set default figure placement to htbp
\makeatletter
\def\fps@figure{htbp}
\makeatother
\usepackage{svg}
\setlength{\emergencystretch}{3em} % prevent overfull lines
\providecommand{\tightlist}{%
  \setlength{\itemsep}{0pt}\setlength{\parskip}{0pt}}
\setcounter{secnumdepth}{-\maxdimen} % remove section numbering
\ifLuaTeX
  \usepackage{selnolig}  % disable illegal ligatures
\fi
\usepackage{bookmark}
\IfFileExists{xurl.sty}{\usepackage{xurl}}{} % add URL line breaks if available
\urlstyle{same}
\hypersetup{
  pdftitle={Starting with Data},
  hidelinks,
  pdfcreator={LaTeX via pandoc}}

\title{Starting with Data}
\author{}
\date{\vspace{-2.5em}}

\begin{document}
\maketitle

The two main goals for this lessons are:

\begin{itemize}
\tightlist
\item
  To make sure that learners are comfortable with working with data
  frames, and can use the bracket notation to select slices/columns.
\item
  To expose learners to factors. Their behavior is not necessarily
  intuitive, and so it is important that they are guided through it the
  first time they are exposed to it. The content of the lesson should be
  enough for learners to avoid common mistakes with them.
\end{itemize}

\begin{itemize}
\tightlist
\item
  Describe what a data frame is.
\item
  Load external data from a .csv file into a data frame.
\item
  Summarize the contents of a data frame.
\item
  Subset values from data frames.
\item
  Describe the difference between a factor and a string.
\item
  Convert between strings and factors.
\item
  Reorder and rename factors.
\item
  Change how character strings are handled in a data frame.
\item
  Examine and change date formats.
\end{itemize}

\begin{itemize}
\tightlist
\item
  What is a data.frame?
\item
  How can I read a complete csv file into R?
\item
  How can I get basic summary information about my dataset?
\item
  How can I change the way R treats strings in my dataset?
\item
  Why would I want strings to be treated differently?
\item
  How are dates represented in R and how can I change the format?
\end{itemize}

\subsection{What are data frames?}\label{what-are-data-frames}

Data frames are the \emph{de facto} data structure for tabular data in
\texttt{R}, and what we use for data processing, statistics, and
plotting.

A data frame is the representation of data in the format of a table
where the columns are vectors that all have the same length. Data frames
are analogous to the more familiar spreadsheet in programs such as
Excel, with one key difference. Because columns are vectors, each column
must contain a single type of data (e.g., characters, integers,
factors). For example, here is a figure depicting a data frame
comprising a numeric, a character, and a logical vector.

\includesvg{fig/data-frame.svg}

Data frames can be created by hand, but most commonly they are generated
by the functions \texttt{read\_csv()} or \texttt{read\_table()}; in
other words, when importing spreadsheets from your hard drive (or the
web). We will now demonstrate how to import tabular data using
\texttt{read\_csv()}.

\subsection{Presentation of the SAFI
Data}\label{presentation-of-the-safi-data}

SAFI (Studying African Farmer-Led Irrigation) is a study looking at
farming and irrigation methods in Tanzania and Mozambique. The survey
data was collected through interviews conducted between November 2016
and June 2017. For this lesson, we will be using a subset of the
available data. For information about the full teaching dataset used in
other lessons in this workshop, see the
\href{https://datacarpentry.org/socialsci-workshop/index.html\#data}{dataset
description}.

We will be using a subset of the cleaned version of the dataset that was
produced through cleaning in OpenRefine (\texttt{data/SAFI\_clean.csv}).
In this dataset, the missing data is encoded as ``NULL'', each row holds
information for a single interview respondent, and the columns
represent:

\begin{longtable}[]{@{}
  >{\raggedright\arraybackslash}p{(\columnwidth - 2\tabcolsep) * \real{0.1351}}
  >{\raggedright\arraybackslash}p{(\columnwidth - 2\tabcolsep) * \real{0.8649}}@{}}
\toprule\noalign{}
\begin{minipage}[b]{\linewidth}\raggedright
column\_name
\end{minipage} & \begin{minipage}[b]{\linewidth}\raggedright
description
\end{minipage} \\
\midrule\noalign{}
\endhead
\bottomrule\noalign{}
\endlastfoot
key\_id & Added to provide a unique Id for each observation. (The
InstanceID field does this as well but it is not as convenient to
use) \\
village & Village name \\
interview\_date & Date of interview \\
no\_membrs & How many members in the household? \\
years\_liv & How many years have you been living in this village or
neighboring village? \\
respondent\_wall\_type & What type of walls does their house have (from
list) \\
rooms & How many rooms in the main house are used for sleeping? \\
memb\_assoc & Are you a member of an irrigation association? \\
affect\_conflicts & Have you been affected by conflicts with other
irrigators in the area? \\
liv\_count & Number of livestock owned. \\
items\_owned & Which of the following items are owned by the household?
(list) \\
no\_meals & How many meals do people in your household normally eat in a
day? \\
months\_lack\_food & Indicate which months, In the last 12 months have
you faced a situation when you did not have enough food to feed the
household? \\
instanceID & Unique identifier for the form data submission \\
\end{longtable}

\subsection{Importing data}\label{importing-data}

You are going to load the data in R's memory using the function
\texttt{read\_csv()} from the \textbf{\texttt{readr}} package, which is
part of the \textbf{\texttt{tidyverse}}; learn more about the
\textbf{\texttt{tidyverse}} collection of packages
\href{https://www.tidyverse.org/}{here}. \textbf{\texttt{readr}} gets
installed as part as the \textbf{\texttt{tidyverse}} installation. When
you load the \textbf{\texttt{tidyverse}} (\texttt{library(tidyverse)}),
the core packages (the packages used in most data analyses) get loaded,
including \textbf{\texttt{readr}}.

Before proceeding, however, this is a good opportunity to talk about
conflicts. Certain packages we load can end up introducing function
names that are already in use by pre-loaded R packages. For instance,
when we load the tidyverse package below, we will introduce two
conflicting functions: \texttt{filter()} and \texttt{lag()}. This
happens because \texttt{filter} and \texttt{lag} are already functions
used by the stats package (already pre-loaded in R). What will happen
now is that if we, for example, call the \texttt{filter()} function, R
will use the \texttt{dplyr::filter()} version and not the
\texttt{stats::filter()} one. This happens because, if conflicted, by
default R uses the function from the most recently loaded package.
Conflicted functions may cause you some trouble in the future, so it is
important that we are aware of them so that we can properly handle them,
if we want.

To do so, we just need the following functions from the conflicted
package:

\begin{itemize}
\tightlist
\item
  \texttt{conflicted::conflict\_scout()}: Shows us any conflicted
  functions.\\
\item
  \texttt{conflict\_prefer("function",\ "package\_prefered")}: Allows us
  to choose the default function we want from now on.
\end{itemize}

It is also important to know that we can, at any time, just call the
function directly from the package we want, such as
\texttt{stats::filter()}.

Even with the use of an RStudio project, it can be difficult to learn
how to specify paths to file locations. Enter the \textbf{here} package!
The here package creates paths relative to the top-level directory (your
RStudio project). These relative paths work \emph{regardless} of where
the associated source file lives inside your project, like analysis
projects with data and reports in different subdirectories. This is an
important contrast to using \texttt{setwd()}, which depends on the way
you order your files on your computer.

\includegraphics[width=1\linewidth,alt={Monsters at a fork in the road, with signs saying here, and not here. One direction, not here, leads to a scary dark forest with spiders and absolute filepaths, while the other leads to a sunny, green meadow, and a city below a rainbow and a world free of absolute filepaths. Art by Allison Horst}]{fig/here_horst}

Image credit: Allison Horst

Before we can use the \texttt{read\_csv()} and \texttt{here()}
functions, we need to load the tidyverse and here packages.

Also, if you recall, the missing data is encoded as ``NULL'' in the
dataset. We'll tell it to the function, so R will automatically convert
all the ``NULL'' entries in the dataset into \texttt{NA}.

\begin{Shaded}
\begin{Highlighting}[]
\FunctionTok{library}\NormalTok{(tidyverse)}
\FunctionTok{library}\NormalTok{(here)}

\NormalTok{interviews }\OtherTok{\textless{}{-}} \FunctionTok{read\_csv}\NormalTok{(}
  \FunctionTok{here}\NormalTok{(}\StringTok{"data"}\NormalTok{, }\StringTok{"SAFI\_clean.csv"}\NormalTok{), }
  \AttributeTok{na =} \StringTok{"NULL"}\NormalTok{)}
\end{Highlighting}
\end{Shaded}

In the above code, we notice the \texttt{here()} function takes folder
and file names as inputs (e.g., \texttt{"data"},
\texttt{"SAFI\_clean.csv"}), each enclosed in quotations (\texttt{""})
and separated by a comma. The \texttt{here()} will accept as many names
as are necessary to navigate to a particular file (e.g.,
\texttt{here("analysis",\ "data",\ "surveys",\ "clean",\ "SAFI\_clean.csv)}).

The \texttt{here()} function can accept the folder and file names in an
alternate format, using a slash (``/'') rather than commas to separate
the names. The two methods are equivalent, so that
\texttt{here("data",\ "SAFI\_clean.csv")} and
\texttt{here("data/SAFI\_clean.csv")} produce the same result. (The
slash is used on all operating systems; backslashes are not used.)

If you were to type in the code above, it is likely that the
\texttt{read.csv()} function would appear in the automatically populated
list of functions. This function is different from the
\texttt{read\_csv()} function, as it is included in the ``base''
packages that come pre-installed with R. Overall, \texttt{read.csv()}
behaves similar to \texttt{read\_csv()}, with a few notable differences.
First, \texttt{read.csv()} coerces column names with spaces and/or
special characters to different names (e.g.~\texttt{interview\ date}
becomes \texttt{interview.date}). Second, \texttt{read.csv()} stores
data as a \texttt{data.frame}, where \texttt{read\_csv()} stores data as
a different kind of data frame called a \texttt{tibble}. We prefer
tibbles because they have nice printing properties among other desirable
qualities. Read more about tibbles
\href{https://tibble.tidyverse.org/}{here}.

The second statement in the code above creates a data frame but doesn't
output any data because, as you might recall, assignments
(\texttt{\textless{}-}) don't display anything. (Note, however, that
\texttt{read\_csv} may show informational text about the data frame that
is created.) If we want to check that our data has been loaded, we can
see the contents of the data frame by typing its name:
\texttt{interviews} in the console.

\begin{Shaded}
\begin{Highlighting}[]
\NormalTok{interviews}
\DocumentationTok{\#\# Try also}
\DocumentationTok{\#\# view(interviews)}
\DocumentationTok{\#\# head(interviews)}
\end{Highlighting}
\end{Shaded}

\begin{verbatim}
## # A tibble: 131 x 14
##    key_ID village  interview_date      no_membrs years_liv respondent_wall_type
##     <dbl> <chr>    <dttm>                  <dbl>     <dbl> <chr>               
##  1      1 God      2016-11-17 00:00:00         3         4 muddaub             
##  2      2 God      2016-11-17 00:00:00         7         9 muddaub             
##  3      3 God      2016-11-17 00:00:00        10        15 burntbricks         
##  4      4 God      2016-11-17 00:00:00         7         6 burntbricks         
##  5      5 God      2016-11-17 00:00:00         7        40 burntbricks         
##  6      6 God      2016-11-17 00:00:00         3         3 muddaub             
##  7      7 God      2016-11-17 00:00:00         6        38 muddaub             
##  8      8 Chirodzo 2016-11-16 00:00:00        12        70 burntbricks         
##  9      9 Chirodzo 2016-11-16 00:00:00         8         6 burntbricks         
## 10     10 Chirodzo 2016-12-16 00:00:00        12        23 burntbricks         
## # i 121 more rows
## # i 8 more variables: rooms <dbl>, memb_assoc <chr>, affect_conflicts <chr>,
## #   liv_count <dbl>, items_owned <chr>, no_meals <dbl>, months_lack_food <chr>,
## #   instanceID <chr>
\end{verbatim}

\subsection{Note}\label{note}

\texttt{read\_csv()} assumes that fields are delimited by commas.
However, in several countries, the comma is used as a decimal separator
and the semicolon (;) is used as a field delimiter. If you want to read
in this type of files in R, you can use the \texttt{read\_csv2}
function. It behaves exactly like \texttt{read\_csv} but uses different
parameters for the decimal and the field separators. If you are working
with another format, they can be both specified by the user. Check out
the help for \texttt{read\_csv()} by typing \texttt{?read\_csv} to learn
more. There is also the \texttt{read\_tsv()} for tab-separated data
files, and \texttt{read\_delim()} allows you to specify more details
about the structure of your file.

Note that \texttt{read\_csv()} actually loads the data as a tibble. A
tibble is an extension of \texttt{R} data frames used by the
\textbf{\texttt{tidyverse}}. When the data is read using
\texttt{read\_csv()}, it is stored in an object of class
\texttt{tbl\_df}, \texttt{tbl}, and \texttt{data.frame}. You can see the
class of an object with

\begin{Shaded}
\begin{Highlighting}[]
\FunctionTok{class}\NormalTok{(interviews)}
\end{Highlighting}
\end{Shaded}

\begin{verbatim}
## [1] "spec_tbl_df" "tbl_df"      "tbl"         "data.frame"
\end{verbatim}

As a \texttt{tibble}, the type of data included in each column is listed
in an abbreviated fashion below the column names. For instance, here
\texttt{key\_ID} is a column of floating point numbers (abbreviated
\texttt{\textless{}dbl\textgreater{}} for the word `double'),
\texttt{village} is a column of characters
(\texttt{\textless{}chr\textgreater{}}) and the \texttt{interview\_date}
is a column in the ``date and time'' format
(\texttt{\textless{}dttm\textgreater{}}).

\subsection{Inspecting data frames}\label{inspecting-data-frames}

When calling a \texttt{tbl\_df} object (like \texttt{interviews} here),
there is already a lot of information about our data frame being
displayed such as the number of rows, the number of columns, the names
of the columns, and as we just saw the class of data stored in each
column. However, there are functions to extract this information from
data frames. Here is a non-exhaustive list of some of these functions.
Let's try them out!

Size:

\begin{itemize}
\tightlist
\item
  \texttt{dim(interviews)} - returns a vector with the number of rows as
  the first element, and the number of columns as the second element
  (the \textbf{dim}ensions of the object)
\item
  \texttt{nrow(interviews)} - returns the number of rows
\item
  \texttt{ncol(interviews)} - returns the number of columns
\end{itemize}

Content:

\begin{itemize}
\tightlist
\item
  \texttt{head(interviews)} - shows the first 6 rows
\item
  \texttt{tail(interviews)} - shows the last 6 rows
\end{itemize}

Names:

\begin{itemize}
\tightlist
\item
  \texttt{names(interviews)} - returns the column names (synonym of
  \texttt{colnames()} for \texttt{data.frame} objects)
\end{itemize}

Summary:

\begin{itemize}
\tightlist
\item
  \texttt{str(interviews)} - structure of the object and information
  about the class, length and content of each column
\item
  \texttt{summary(interviews)} - summary statistics for each column
\item
  \texttt{glimpse(interviews)} - returns the number of columns and rows
  of the tibble, the names and class of each column, and previews as
  many values will fit on the screen. Unlike the other inspecting
  functions listed above, \texttt{glimpse()} is not a ``base R''
  function so you need to have the \texttt{dplyr} or \texttt{tibble}
  packages loaded to be able to execute it.
\end{itemize}

Note: most of these functions are ``generic.'' They can be used on other
types of objects besides data frames or tibbles.

\subsection{Subsetting data frames}\label{subsetting-data-frames}

Our \texttt{interviews} data frame has rows and columns (it has 2
dimensions). In practice, we may not need the entire data frame; for
instance, we may only be interested in a subset of the observations (the
rows) or a particular set of variables (the columns). If we want to
access some specific data from it, we need to specify the
``coordinates'' (i.e., indices) we want from it. Row numbers come first,
followed by column numbers.

\subsection{Tip}\label{tip}

Subsetting a \texttt{tibble} with \texttt{{[}} always results in a
\texttt{tibble}. However, note this is not true in general for data
frames, so be careful! Different ways of specifying these coordinates
can lead to results with different classes. This is covered in the
Software Carpentry lesson
\href{https://swcarpentry.github.io/r-novice-gapminder/}{R for
Reproducible Scientific Analysis}.

\begin{Shaded}
\begin{Highlighting}[]
\DocumentationTok{\#\# first element in the first column of the tibble}
\NormalTok{interviews[}\DecValTok{1}\NormalTok{, }\DecValTok{1}\NormalTok{]}
\end{Highlighting}
\end{Shaded}

\begin{verbatim}
## # A tibble: 1 x 1
##   key_ID
##    <dbl>
## 1      1
\end{verbatim}

\begin{Shaded}
\begin{Highlighting}[]
\DocumentationTok{\#\# first element in the 6th column of the tibble }
\NormalTok{interviews[}\DecValTok{1}\NormalTok{, }\DecValTok{6}\NormalTok{]}
\end{Highlighting}
\end{Shaded}

\begin{verbatim}
## # A tibble: 1 x 1
##   respondent_wall_type
##   <chr>               
## 1 muddaub
\end{verbatim}

\begin{Shaded}
\begin{Highlighting}[]
\DocumentationTok{\#\# first column of the tibble (as a vector)}
\NormalTok{interviews[[}\DecValTok{1}\NormalTok{]]}
\end{Highlighting}
\end{Shaded}

\begin{verbatim}
##   [1]   1   2   3   4   5   6   7   8   9  10  11  12  13  14  15  16  17  18
##  [19]  19  20  21  22  23  24  25  26  27  28  29  30  31  32  33  34  35  36
##  [37]  37  38  39  40  41  42  43  44  45  46  47  48  49  50  51  52  53  54
##  [55]  55  56  57  58  59  60  61  62  63  64  65  66  67  68  69  70  71 127
##  [73] 133 152 153 155 178 177 180 181 182 186 187 195 196 197 198 201 202  72
##  [91]  73  76  83  85  89 101 103 102  78  80 104 105 106 109 110 113 118 125
## [109] 119 115 108 116 117 144 143 150 159 160 165 166 167 174 175 189 191 192
## [127] 126 193 194 199 200
\end{verbatim}

\begin{Shaded}
\begin{Highlighting}[]
\DocumentationTok{\#\# first column of the tibble}
\NormalTok{interviews[}\DecValTok{1}\NormalTok{]}
\end{Highlighting}
\end{Shaded}

\begin{verbatim}
## # A tibble: 131 x 1
##    key_ID
##     <dbl>
##  1      1
##  2      2
##  3      3
##  4      4
##  5      5
##  6      6
##  7      7
##  8      8
##  9      9
## 10     10
## # i 121 more rows
\end{verbatim}

\begin{Shaded}
\begin{Highlighting}[]
\DocumentationTok{\#\# first three elements in the 7th column of the tibble}
\NormalTok{interviews[}\DecValTok{1}\SpecialCharTok{:}\DecValTok{3}\NormalTok{, }\DecValTok{7}\NormalTok{]}
\end{Highlighting}
\end{Shaded}

\begin{verbatim}
## # A tibble: 3 x 1
##   rooms
##   <dbl>
## 1     1
## 2     1
## 3     1
\end{verbatim}

\begin{Shaded}
\begin{Highlighting}[]
\DocumentationTok{\#\# the 3rd row of the tibble}
\NormalTok{interviews[}\DecValTok{3}\NormalTok{, ]}
\end{Highlighting}
\end{Shaded}

\begin{verbatim}
## # A tibble: 1 x 14
##   key_ID village interview_date      no_membrs years_liv respondent_wall_type
##    <dbl> <chr>   <dttm>                  <dbl>     <dbl> <chr>               
## 1      3 God     2016-11-17 00:00:00        10        15 burntbricks         
## # i 8 more variables: rooms <dbl>, memb_assoc <chr>, affect_conflicts <chr>,
## #   liv_count <dbl>, items_owned <chr>, no_meals <dbl>, months_lack_food <chr>,
## #   instanceID <chr>
\end{verbatim}

\begin{Shaded}
\begin{Highlighting}[]
\DocumentationTok{\#\# equivalent to head\_interviews \textless{}{-} head(interviews)}
\NormalTok{head\_interviews }\OtherTok{\textless{}{-}}\NormalTok{ interviews[}\DecValTok{1}\SpecialCharTok{:}\DecValTok{6}\NormalTok{, ]}
\end{Highlighting}
\end{Shaded}

\texttt{:} is a special function that creates numeric vectors of
integers in increasing or decreasing order, test \texttt{1:10} and
\texttt{10:1} for instance.

You can also exclude certain indices of a data frame using the
``\texttt{-}'' sign:

\begin{Shaded}
\begin{Highlighting}[]
\NormalTok{interviews[, }\SpecialCharTok{{-}}\DecValTok{1}\NormalTok{]          }\CommentTok{\# The whole tibble, except the first column}
\end{Highlighting}
\end{Shaded}

\begin{verbatim}
## # A tibble: 131 x 13
##    village  interview_date      no_membrs years_liv respondent_wall_type rooms
##    <chr>    <dttm>                  <dbl>     <dbl> <chr>                <dbl>
##  1 God      2016-11-17 00:00:00         3         4 muddaub                  1
##  2 God      2016-11-17 00:00:00         7         9 muddaub                  1
##  3 God      2016-11-17 00:00:00        10        15 burntbricks              1
##  4 God      2016-11-17 00:00:00         7         6 burntbricks              1
##  5 God      2016-11-17 00:00:00         7        40 burntbricks              1
##  6 God      2016-11-17 00:00:00         3         3 muddaub                  1
##  7 God      2016-11-17 00:00:00         6        38 muddaub                  1
##  8 Chirodzo 2016-11-16 00:00:00        12        70 burntbricks              3
##  9 Chirodzo 2016-11-16 00:00:00         8         6 burntbricks              1
## 10 Chirodzo 2016-12-16 00:00:00        12        23 burntbricks              5
## # i 121 more rows
## # i 7 more variables: memb_assoc <chr>, affect_conflicts <chr>,
## #   liv_count <dbl>, items_owned <chr>, no_meals <dbl>, months_lack_food <chr>,
## #   instanceID <chr>
\end{verbatim}

\begin{Shaded}
\begin{Highlighting}[]
\NormalTok{interviews[}\SpecialCharTok{{-}}\FunctionTok{c}\NormalTok{(}\DecValTok{7}\SpecialCharTok{:}\DecValTok{131}\NormalTok{), ]   }\CommentTok{\# Equivalent to head(interviews)}
\end{Highlighting}
\end{Shaded}

\begin{verbatim}
## # A tibble: 6 x 14
##   key_ID village interview_date      no_membrs years_liv respondent_wall_type
##    <dbl> <chr>   <dttm>                  <dbl>     <dbl> <chr>               
## 1      1 God     2016-11-17 00:00:00         3         4 muddaub             
## 2      2 God     2016-11-17 00:00:00         7         9 muddaub             
## 3      3 God     2016-11-17 00:00:00        10        15 burntbricks         
## 4      4 God     2016-11-17 00:00:00         7         6 burntbricks         
## 5      5 God     2016-11-17 00:00:00         7        40 burntbricks         
## 6      6 God     2016-11-17 00:00:00         3         3 muddaub             
## # i 8 more variables: rooms <dbl>, memb_assoc <chr>, affect_conflicts <chr>,
## #   liv_count <dbl>, items_owned <chr>, no_meals <dbl>, months_lack_food <chr>,
## #   instanceID <chr>
\end{verbatim}

\texttt{tibble}s can be subset by calling indices (as shown previously),
but also by calling their column names directly:

\begin{Shaded}
\begin{Highlighting}[]
\NormalTok{interviews[}\StringTok{"village"}\NormalTok{]       }\CommentTok{\# Result is a tibble}

\NormalTok{interviews[, }\StringTok{"village"}\NormalTok{]     }\CommentTok{\# Result is a tibble}

\NormalTok{interviews[[}\StringTok{"village"}\NormalTok{]]     }\CommentTok{\# Result is a vector}

\NormalTok{interviews}\SpecialCharTok{$}\NormalTok{village          }\CommentTok{\# Result is a vector}
\end{Highlighting}
\end{Shaded}

In RStudio, you can use the autocompletion feature to get the full and
correct names of the columns.

\subsection{Exercise}\label{exercise}

\begin{enumerate}
\def\labelenumi{\arabic{enumi}.}
\tightlist
\item
  Create a tibble (\texttt{interviews\_100}) containing only the data in
  row 100 of the \texttt{interviews} dataset.
\end{enumerate}

Now, continue using \texttt{interviews} for each of the following
activities:

\begin{enumerate}
\def\labelenumi{\arabic{enumi}.}
\setcounter{enumi}{1}
\tightlist
\item
  Notice how \texttt{nrow()} gave you the number of rows in the tibble?
\end{enumerate}

\begin{itemize}
\tightlist
\item
  Use that number to pull out just that last row in the tibble.
\item
  Compare that with what you see as the last row using \texttt{tail()}
  to make sure it's meeting expectations.
\item
  Pull out that last row using \texttt{nrow()} instead of the row
  number.
\item
  Create a new tibble (\texttt{interviews\_last}) from that last row.
\end{itemize}

\begin{enumerate}
\def\labelenumi{\arabic{enumi}.}
\setcounter{enumi}{2}
\item
  Using the number of rows in the interviews dataset that you found in
  question 2, extract the row that is in the middle of the dataset.
  Store the content of this middle row in an object named
  \texttt{interviews\_middle}. (hint: This dataset has an odd number of
  rows, so finding the middle is a bit trickier than dividing n\_rows by
  2. Use the median( ) function and what you've learned about sequences
  in R to extract the middle row!
\item
  Combine \texttt{nrow()} with the \texttt{-} notation above to
  reproduce the behavior of \texttt{head(interviews)}, keeping just the
  first through 6th rows of the interviews dataset.
\end{enumerate}

\subsection{Solution}\label{solution}

\begin{Shaded}
\begin{Highlighting}[]
\DocumentationTok{\#\# 1.}
\NormalTok{interviews\_100 }\OtherTok{\textless{}{-}}\NormalTok{ interviews[}\DecValTok{100}\NormalTok{, ]}
\DocumentationTok{\#\# 2.}
\CommentTok{\# Saving \textasciigrave{}n\_rows\textasciigrave{} to improve readability and reduce duplication}
\NormalTok{n\_rows }\OtherTok{\textless{}{-}} \FunctionTok{nrow}\NormalTok{(interviews)}
\NormalTok{interviews\_last }\OtherTok{\textless{}{-}}\NormalTok{ interviews[n\_rows, ]}
\DocumentationTok{\#\# 3.}
\NormalTok{interviews\_middle }\OtherTok{\textless{}{-}}\NormalTok{ interviews[}\FunctionTok{median}\NormalTok{(}\DecValTok{1}\SpecialCharTok{:}\NormalTok{n\_rows), ]}
\DocumentationTok{\#\# 4.}
\NormalTok{interviews\_head }\OtherTok{\textless{}{-}}\NormalTok{ interviews[}\SpecialCharTok{{-}}\NormalTok{(}\DecValTok{7}\SpecialCharTok{:}\NormalTok{n\_rows), ]}
\end{Highlighting}
\end{Shaded}

\subsection{Factors}\label{factors}

R has a special data class, called factor, to deal with categorical data
that you may encounter when creating plots or doing statistical
analyses. Factors are very useful and actually contribute to making R
particularly well suited to working with data. So we are going to spend
a little time introducing them.

Factors represent categorical data. They are stored as integers
associated with labels and they can be ordered (ordinal) or unordered
(nominal). Factors create a structured relation between the different
levels (values) of a categorical variable, such as days of the week or
responses to a question in a survey. This can make it easier to see how
one element relates to the other elements in a column. While factors
look (and often behave) like character vectors, they are actually
treated as integer vectors by \texttt{R}. So you need to be very careful
when treating them as strings.

Once created, factors can only contain a pre-defined set of values,
known as \emph{levels}. By default, R always sorts levels in
alphabetical order. For instance, if you have a factor with 2 levels:

\begin{Shaded}
\begin{Highlighting}[]
\NormalTok{respondent\_floor\_type }\OtherTok{\textless{}{-}} \FunctionTok{factor}\NormalTok{(}\FunctionTok{c}\NormalTok{(}\StringTok{"earth"}\NormalTok{, }\StringTok{"cement"}\NormalTok{, }\StringTok{"cement"}\NormalTok{, }\StringTok{"earth"}\NormalTok{))}
\end{Highlighting}
\end{Shaded}

R will assign \texttt{1} to the level \texttt{"cement"} and \texttt{2}
to the level \texttt{"earth"} (because \texttt{c} comes before
\texttt{e}, even though the first element in this vector is
\texttt{"earth"}). You can see this by using the function
\texttt{levels()} and you can find the number of levels using
\texttt{nlevels()}:

\begin{Shaded}
\begin{Highlighting}[]
\FunctionTok{levels}\NormalTok{(respondent\_floor\_type)}
\end{Highlighting}
\end{Shaded}

\begin{verbatim}
## [1] "cement" "earth"
\end{verbatim}

\begin{Shaded}
\begin{Highlighting}[]
\FunctionTok{nlevels}\NormalTok{(respondent\_floor\_type)}
\end{Highlighting}
\end{Shaded}

\begin{verbatim}
## [1] 2
\end{verbatim}

Sometimes, the order of the factors does not matter. Other times you
might want to specify the order because it is meaningful (e.g., ``low'',
``medium'', ``high''). It may improve your visualization, or it may be
required by a particular type of analysis. Here, one way to reorder our
levels in the \texttt{respondent\_floor\_type} vector would be:

\begin{Shaded}
\begin{Highlighting}[]
\NormalTok{respondent\_floor\_type }\CommentTok{\# current order}
\end{Highlighting}
\end{Shaded}

\begin{verbatim}
## [1] earth  cement cement earth 
## Levels: cement earth
\end{verbatim}

\begin{Shaded}
\begin{Highlighting}[]
\NormalTok{respondent\_floor\_type }\OtherTok{\textless{}{-}} \FunctionTok{factor}\NormalTok{(respondent\_floor\_type, }
                                \AttributeTok{levels =} \FunctionTok{c}\NormalTok{(}\StringTok{"earth"}\NormalTok{, }\StringTok{"cement"}\NormalTok{))}

\NormalTok{respondent\_floor\_type }\CommentTok{\# after re{-}ordering}
\end{Highlighting}
\end{Shaded}

\begin{verbatim}
## [1] earth  cement cement earth 
## Levels: earth cement
\end{verbatim}

In R's memory, these factors are represented by integers (1, 2), but are
more informative than integers because factors are self describing:
\texttt{"cement"}, \texttt{"earth"} is more descriptive than \texttt{1},
and \texttt{2}. Which one is ``earth''? You wouldn't be able to tell
just from the integer data. Factors, on the other hand, have this
information built in. It is particularly helpful when there are many
levels. It also makes renaming levels easier. Let's say we made a
mistake and need to recode ``cement'' to ``brick''. We can do this using
the \texttt{fct\_recode()} function from the \textbf{\texttt{forcats}}
package (included in the \textbf{\texttt{tidyverse}}) which provides
some extra tools to work with factors.

\begin{Shaded}
\begin{Highlighting}[]
\FunctionTok{levels}\NormalTok{(respondent\_floor\_type)}
\end{Highlighting}
\end{Shaded}

\begin{verbatim}
## [1] "earth"  "cement"
\end{verbatim}

\begin{Shaded}
\begin{Highlighting}[]
\NormalTok{respondent\_floor\_type }\OtherTok{\textless{}{-}} \FunctionTok{fct\_recode}\NormalTok{(respondent\_floor\_type, }\AttributeTok{brick =} \StringTok{"cement"}\NormalTok{)}

\DocumentationTok{\#\# as an alternative, we could change the "cement" level directly using the}
\DocumentationTok{\#\# levels() function, but we have to remember that "cement" is the second level}
\CommentTok{\# levels(respondent\_floor\_type)[2] \textless{}{-} "brick"}

\FunctionTok{levels}\NormalTok{(respondent\_floor\_type)}
\end{Highlighting}
\end{Shaded}

\begin{verbatim}
## [1] "earth" "brick"
\end{verbatim}

\begin{Shaded}
\begin{Highlighting}[]
\NormalTok{respondent\_floor\_type}
\end{Highlighting}
\end{Shaded}

\begin{verbatim}
## [1] earth brick brick earth
## Levels: earth brick
\end{verbatim}

So far, your factor is unordered, like a nominal variable. R does not
know the difference between a nominal and an ordinal variable. You make
your factor an ordered factor by using the \texttt{ordered=TRUE} option
inside your factor function. Note how the reported levels changed from
the unordered factor above to the ordered version below. Ordered levels
use the less than sign \texttt{\textless{}} to denote level ranking.

\begin{Shaded}
\begin{Highlighting}[]
\NormalTok{respondent\_floor\_type\_ordered }\OtherTok{\textless{}{-}} \FunctionTok{factor}\NormalTok{(respondent\_floor\_type, }
                                        \AttributeTok{ordered =} \ConstantTok{TRUE}\NormalTok{)}

\NormalTok{respondent\_floor\_type\_ordered }\CommentTok{\# after setting as ordered factor}
\end{Highlighting}
\end{Shaded}

\begin{verbatim}
## [1] earth brick brick earth
## Levels: earth < brick
\end{verbatim}

\subsubsection{Converting factors}\label{converting-factors}

If you need to convert a factor to a character vector, you use
\texttt{as.character(x)}.

\begin{Shaded}
\begin{Highlighting}[]
\FunctionTok{as.character}\NormalTok{(respondent\_floor\_type)}
\end{Highlighting}
\end{Shaded}

\begin{verbatim}
## [1] "earth" "brick" "brick" "earth"
\end{verbatim}

Converting factors where the levels appear as numbers (such as
concentration levels, or years) to a numeric vector is a little
trickier. The \texttt{as.numeric()} function returns the index values of
the factor, not its levels, so it will result in an entirely new (and
unwanted in this case) set of numbers. One method to avoid this is to
convert factors to characters, and then to numbers. Another method is to
use the \texttt{levels()} function. Compare:

\begin{Shaded}
\begin{Highlighting}[]
\NormalTok{year\_fct }\OtherTok{\textless{}{-}} \FunctionTok{factor}\NormalTok{(}\FunctionTok{c}\NormalTok{(}\DecValTok{1990}\NormalTok{, }\DecValTok{1983}\NormalTok{, }\DecValTok{1977}\NormalTok{, }\DecValTok{1998}\NormalTok{, }\DecValTok{1990}\NormalTok{))}

\FunctionTok{as.numeric}\NormalTok{(year\_fct)                     }\CommentTok{\# Wrong! And there is no warning...}
\end{Highlighting}
\end{Shaded}

\begin{verbatim}
## [1] 3 2 1 4 3
\end{verbatim}

\begin{Shaded}
\begin{Highlighting}[]
\FunctionTok{as.numeric}\NormalTok{(}\FunctionTok{as.character}\NormalTok{(year\_fct))       }\CommentTok{\# Works...}
\end{Highlighting}
\end{Shaded}

\begin{verbatim}
## [1] 1990 1983 1977 1998 1990
\end{verbatim}

\begin{Shaded}
\begin{Highlighting}[]
\FunctionTok{as.numeric}\NormalTok{(}\FunctionTok{levels}\NormalTok{(year\_fct))[year\_fct]   }\CommentTok{\# The recommended way.}
\end{Highlighting}
\end{Shaded}

\begin{verbatim}
## [1] 1990 1983 1977 1998 1990
\end{verbatim}

Notice that in the recommended \texttt{levels()} approach, three
important steps occur:

\begin{itemize}
\tightlist
\item
  We obtain all the factor levels using \texttt{levels(year\_fct)}
\item
  We convert these levels to numeric values using
  \texttt{as.numeric(levels(year\_fct))}
\item
  We then access these numeric values using the underlying integers of
  the vector \texttt{year\_fct} inside the square brackets
\end{itemize}

\subsubsection{Renaming factors}\label{renaming-factors}

When your data is stored as a factor, you can use the \texttt{plot()}
function to get a quick glance at the number of observations represented
by each factor level. Let's extract the \texttt{memb\_assoc} column from
our data frame, convert it into a factor, and use it to look at the
number of interview respondents who were or were not members of an
irrigation association:

\begin{Shaded}
\begin{Highlighting}[]
\DocumentationTok{\#\# create a vector from the data frame column "memb\_assoc"}
\NormalTok{memb\_assoc }\OtherTok{\textless{}{-}}\NormalTok{ interviews}\SpecialCharTok{$}\NormalTok{memb\_assoc}

\DocumentationTok{\#\# convert it into a factor}
\NormalTok{memb\_assoc }\OtherTok{\textless{}{-}} \FunctionTok{as.factor}\NormalTok{(memb\_assoc)}

\DocumentationTok{\#\# let\textquotesingle{}s see what it looks like}
\NormalTok{memb\_assoc}
\end{Highlighting}
\end{Shaded}

\begin{verbatim}
##   [1] <NA> yes  <NA> <NA> <NA> <NA> no   yes  no   no   <NA> yes  no   <NA> yes 
##  [16] <NA> <NA> <NA> <NA> <NA> no   <NA> <NA> no   no   no   <NA> no   yes  <NA>
##  [31] <NA> yes  no   yes  yes  yes  <NA> yes  <NA> yes  <NA> no   no   <NA> no  
##  [46] no   yes  <NA> <NA> yes  <NA> no   yes  no   <NA> yes  no   no   <NA> no  
##  [61] yes  <NA> <NA> <NA> no   yes  no   no   no   no   yes  <NA> no   yes  <NA>
##  [76] <NA> yes  no   no   yes  no   no   yes  no   yes  no   no   <NA> yes  yes 
##  [91] yes  yes  yes  no   no   no   no   yes  no   no   yes  yes  no   <NA> no  
## [106] no   <NA> no   no   <NA> no   <NA> <NA> no   no   no   no   yes  no   no  
## [121] no   no   no   no   no   no   no   no   no   yes  <NA>
## Levels: no yes
\end{verbatim}

\begin{Shaded}
\begin{Highlighting}[]
\DocumentationTok{\#\# bar plot of the number of interview respondents who were}
\DocumentationTok{\#\# members of irrigation association:}
\FunctionTok{plot}\NormalTok{(memb\_assoc)}
\end{Highlighting}
\end{Shaded}

\includegraphics[alt={Yes/no bar graph showing number of individuals who are members of irrigation association}]{02-starting-with-data-exported_files/figure-latex/factor-plot-default-order-1}

Looking at the plot compared to the output of the vector, we can see
that in addition to ``no''s and ``yes''s, there are some respondents for
whom the information about whether they were part of an irrigation
association hasn't been recorded, and encoded as missing data. These
respondents do not appear on the plot. Let's encode them differently so
they can be counted and visualized in our plot.

\begin{Shaded}
\begin{Highlighting}[]
\DocumentationTok{\#\# Let\textquotesingle{}s recreate the vector from the data frame column "memb\_assoc"}
\NormalTok{memb\_assoc }\OtherTok{\textless{}{-}}\NormalTok{ interviews}\SpecialCharTok{$}\NormalTok{memb\_assoc}

\DocumentationTok{\#\# replace the missing data with "undetermined"}
\NormalTok{memb\_assoc[}\FunctionTok{is.na}\NormalTok{(memb\_assoc)] }\OtherTok{\textless{}{-}} \StringTok{"undetermined"}

\DocumentationTok{\#\# convert it into a factor}
\NormalTok{memb\_assoc }\OtherTok{\textless{}{-}} \FunctionTok{as.factor}\NormalTok{(memb\_assoc)}

\DocumentationTok{\#\# let\textquotesingle{}s see what it looks like}
\NormalTok{memb\_assoc}
\end{Highlighting}
\end{Shaded}

\begin{verbatim}
##   [1] undetermined yes          undetermined undetermined undetermined
##   [6] undetermined no           yes          no           no          
##  [11] undetermined yes          no           undetermined yes         
##  [16] undetermined undetermined undetermined undetermined undetermined
##  [21] no           undetermined undetermined no           no          
##  [26] no           undetermined no           yes          undetermined
##  [31] undetermined yes          no           yes          yes         
##  [36] yes          undetermined yes          undetermined yes         
##  [41] undetermined no           no           undetermined no          
##  [46] no           yes          undetermined undetermined yes         
##  [51] undetermined no           yes          no           undetermined
##  [56] yes          no           no           undetermined no          
##  [61] yes          undetermined undetermined undetermined no          
##  [66] yes          no           no           no           no          
##  [71] yes          undetermined no           yes          undetermined
##  [76] undetermined yes          no           no           yes         
##  [81] no           no           yes          no           yes         
##  [86] no           no           undetermined yes          yes         
##  [91] yes          yes          yes          no           no          
##  [96] no           no           yes          no           no          
## [101] yes          yes          no           undetermined no          
## [106] no           undetermined no           no           undetermined
## [111] no           undetermined undetermined no           no          
## [116] no           no           yes          no           no          
## [121] no           no           no           no           no          
## [126] no           no           no           no           yes         
## [131] undetermined
## Levels: no undetermined yes
\end{verbatim}

\begin{Shaded}
\begin{Highlighting}[]
\DocumentationTok{\#\# bar plot of the number of interview respondents who were}
\DocumentationTok{\#\# members of irrigation association:}
\FunctionTok{plot}\NormalTok{(memb\_assoc)}
\end{Highlighting}
\end{Shaded}

\includegraphics[alt={Bar plot of association membership, showing missing responses.}]{02-starting-with-data-exported_files/figure-latex/factor-plot-reorder-1}

\subsection{Exercise}\label{exercise-1}

\begin{itemize}
\item
  Rename the levels of the factor to have the first letter in uppercase:
  ``No'',``Undetermined'', and ``Yes''.
\item
  Now that we have renamed the factor level to ``Undetermined'', can you
  recreate the barplot such that ``Undetermined'' is last (after
  ``Yes'')?
\end{itemize}

\subsection{Solution}\label{solution-1}

\begin{Shaded}
\begin{Highlighting}[]
\DocumentationTok{\#\# Rename levels.}
\NormalTok{memb\_assoc }\OtherTok{\textless{}{-}} \FunctionTok{fct\_recode}\NormalTok{(memb\_assoc, }\AttributeTok{No =} \StringTok{"no"}\NormalTok{,}
                         \AttributeTok{Undetermined =} \StringTok{"undetermined"}\NormalTok{, }\AttributeTok{Yes =} \StringTok{"yes"}\NormalTok{)}
\DocumentationTok{\#\# Reorder levels. Note we need to use the new level names.}
\NormalTok{memb\_assoc }\OtherTok{\textless{}{-}} \FunctionTok{factor}\NormalTok{(memb\_assoc, }\AttributeTok{levels =} \FunctionTok{c}\NormalTok{(}\StringTok{"No"}\NormalTok{, }\StringTok{"Yes"}\NormalTok{, }\StringTok{"Undetermined"}\NormalTok{))}
\FunctionTok{plot}\NormalTok{(memb\_assoc)}
\end{Highlighting}
\end{Shaded}

\includegraphics[alt={bar graph showing number of individuals who are members of irrigation association, including undetermined option}]{02-starting-with-data-exported_files/figure-latex/factor-plot-exercise-1}

\subsection{Formatting Dates}\label{formatting-dates}

One of the most common issues that new (and experienced!) R users have
is converting date and time information into a variable that is
appropriate and usable during analyses. A best practice for dealing with
date data is to ensure that each component of your date is available as
a separate variable. In our dataset, we have a column
\texttt{interview\_date} which contains information about the year,
month, and day that the interview was conducted. Let's convert those
dates into three separate columns.

\begin{Shaded}
\begin{Highlighting}[]
\FunctionTok{str}\NormalTok{(interviews)}
\end{Highlighting}
\end{Shaded}

We are going to use the package \textbf{\texttt{lubridate}}, , which is
included in the \textbf{\texttt{tidyverse}} installation and should be
loaded by default. However, if we deal with older versions of tidyverse
(2022 and ealier), we can manually load it by typing
\texttt{library(lubridate)}.

If necessary, start by loading the required package:

\begin{Shaded}
\begin{Highlighting}[]
\FunctionTok{library}\NormalTok{(lubridate)}
\end{Highlighting}
\end{Shaded}

The lubridate function \texttt{ymd()} takes a vector representing year,
month, and day, and converts it to a \texttt{Date} vector. \texttt{Date}
is a class of data recognized by R as being a date and can be
manipulated as such. The argument that the function requires is
flexible, but, as a best practice, is a character vector formatted as
``YYYY-MM-DD''.

Let's extract our \texttt{interview\_date} column and inspect the
structure:

\begin{Shaded}
\begin{Highlighting}[]
\NormalTok{dates }\OtherTok{\textless{}{-}}\NormalTok{ interviews}\SpecialCharTok{$}\NormalTok{interview\_date}
\FunctionTok{str}\NormalTok{(dates)}
\end{Highlighting}
\end{Shaded}

\begin{verbatim}
##  POSIXct[1:131], format: "2016-11-17" "2016-11-17" "2016-11-17" "2016-11-17" "2016-11-17" ...
\end{verbatim}

When we imported the data in R, \texttt{read\_csv()} recognized that
this column contained date information. We can now use the
\texttt{day()}, \texttt{month()} and \texttt{year()} functions to
extract this information from the date, and create new columns in our
data frame to store it:

\begin{Shaded}
\begin{Highlighting}[]
\NormalTok{interviews}\SpecialCharTok{$}\NormalTok{day }\OtherTok{\textless{}{-}} \FunctionTok{day}\NormalTok{(dates)}
\NormalTok{interviews}\SpecialCharTok{$}\NormalTok{month }\OtherTok{\textless{}{-}} \FunctionTok{month}\NormalTok{(dates)}
\NormalTok{interviews}\SpecialCharTok{$}\NormalTok{year }\OtherTok{\textless{}{-}} \FunctionTok{year}\NormalTok{(dates)}
\NormalTok{interviews}
\end{Highlighting}
\end{Shaded}

\begin{verbatim}
## # A tibble: 131 x 17
##    key_ID village  interview_date      no_membrs years_liv respondent_wall_type
##     <dbl> <chr>    <dttm>                  <dbl>     <dbl> <chr>               
##  1      1 God      2016-11-17 00:00:00         3         4 muddaub             
##  2      2 God      2016-11-17 00:00:00         7         9 muddaub             
##  3      3 God      2016-11-17 00:00:00        10        15 burntbricks         
##  4      4 God      2016-11-17 00:00:00         7         6 burntbricks         
##  5      5 God      2016-11-17 00:00:00         7        40 burntbricks         
##  6      6 God      2016-11-17 00:00:00         3         3 muddaub             
##  7      7 God      2016-11-17 00:00:00         6        38 muddaub             
##  8      8 Chirodzo 2016-11-16 00:00:00        12        70 burntbricks         
##  9      9 Chirodzo 2016-11-16 00:00:00         8         6 burntbricks         
## 10     10 Chirodzo 2016-12-16 00:00:00        12        23 burntbricks         
## # i 121 more rows
## # i 11 more variables: rooms <dbl>, memb_assoc <chr>, affect_conflicts <chr>,
## #   liv_count <dbl>, items_owned <chr>, no_meals <dbl>, months_lack_food <chr>,
## #   instanceID <chr>, day <int>, month <dbl>, year <dbl>
\end{verbatim}

Notice the three new columns at the end of our data frame.

In our example above, the \texttt{interview\_date} column was read in
correctly as a \texttt{Date} variable but generally that is not the
case. Date columns are often read in as \texttt{character} variables and
one can use the \texttt{as\_date()} function to convert them to the
appropriate \texttt{Date/POSIXct}format.

Let's say we have a vector of dates in character format:

\begin{Shaded}
\begin{Highlighting}[]
\NormalTok{char\_dates }\OtherTok{\textless{}{-}} \FunctionTok{c}\NormalTok{(}\StringTok{"7/31/2012"}\NormalTok{, }\StringTok{"8/9/2014"}\NormalTok{, }\StringTok{"4/30/2016"}\NormalTok{)}
\FunctionTok{str}\NormalTok{(char\_dates)}
\end{Highlighting}
\end{Shaded}

\begin{verbatim}
##  chr [1:3] "7/31/2012" "8/9/2014" "4/30/2016"
\end{verbatim}

We can convert this vector to dates as :

\begin{Shaded}
\begin{Highlighting}[]
\FunctionTok{as\_date}\NormalTok{(char\_dates, }\AttributeTok{format =} \StringTok{"\%m/\%d/\%Y"}\NormalTok{)}
\end{Highlighting}
\end{Shaded}

\begin{verbatim}
## [1] "2012-07-31" "2014-08-09" "2016-04-30"
\end{verbatim}

Argument \texttt{format} tells the function the order to parse the
characters and identify the month, day and year. The format above is the
equivalent of mm/dd/yyyy. A wrong format can lead to parsing errors or
incorrect results.

For example, observe what happens when we use a lower case y instead of
upper case Y for the year.

\begin{Shaded}
\begin{Highlighting}[]
\FunctionTok{as\_date}\NormalTok{(char\_dates, }\AttributeTok{format =} \StringTok{"\%m/\%d/\%y"}\NormalTok{)}
\end{Highlighting}
\end{Shaded}

\begin{verbatim}
## Warning: 3 failed to parse.
\end{verbatim}

\begin{verbatim}
## [1] NA NA NA
\end{verbatim}

Here, the \texttt{\%y} part of the format stands for a two-digit year
instead of a four-digit year, and this leads to parsing errors.

Or in the following example, observe what happens when the month and day
elements of the format are switched.

\begin{Shaded}
\begin{Highlighting}[]
\FunctionTok{as\_date}\NormalTok{(char\_dates, }\AttributeTok{format =} \StringTok{"\%d/\%m/\%y"}\NormalTok{)}
\end{Highlighting}
\end{Shaded}

\begin{verbatim}
## Warning: 3 failed to parse.
\end{verbatim}

\begin{verbatim}
## [1] NA NA NA
\end{verbatim}

Since there is no month numbered 30 or 31, the first and third dates
cannot be parsed.

We can also use functions \texttt{ymd()}, \texttt{mdy()} or
\texttt{dmy()} to convert character variables to date.

\begin{Shaded}
\begin{Highlighting}[]
\FunctionTok{mdy}\NormalTok{(char\_dates)}
\end{Highlighting}
\end{Shaded}

\begin{verbatim}
## [1] "2012-07-31" "2014-08-09" "2016-04-30"
\end{verbatim}

\begin{itemize}
\tightlist
\item
  Use read\_csv to read tabular data in R.
\item
  Use factors to represent categorical data in R.
\end{itemize}

\end{document}
